\subsection{Consultative and advisory functions to other laboratories and agencies, especially Air Force and other DoD laboratories. Provide factual information about the subject matter, institutions, locations, dates, and names(s) of principal individuals involved}

\subsubsection{Janos Sztipanovits}

\emph{Air Force Scientific Advisory Board}

Vice-chair of the S\&T review panel of AFRL/RI in Oct. 2007.

Executive review of the AFRL Flight Critical Systems Software Initiative (FCSSI): AFRL, Air Vehicles Directorate, 30 Apr 2008.  David Homan, AFRL, Control Systems Development and Applications.

Study Chair of the AF SAB FY08 Study on ``Defending and Operating in a Cyber Contested Environment.''

Member of the AF SAB FY 10 Study on ``Operating the Next-Generation Unmanned Aerial Systems for Irregular Warfare.''

\emph{National Aeronautics and Space Administration}

Member of the NASA Advisory Council - Exploration Subcommittee on Avionics, Software and Cyber-security. 2009-2012.


\subsubsection{Shankar Sastry}

Technical Advisory Group, President's Council on Science and Technology, 2006-2007.

Member of the AF SAB 9/03-9/06 responsible for reviews of information technology programs in the Air Force.


                 \subsubsection{Edward A. Lee} 
                 \emph{Air Force Research Laboratory, AFRL/RIEA, Rome, NY}

                 Brian Romano USAF AFMC /AFRL/RIEF, Brian.Romano@rl.af.mil

                 The objective of the Extensible Modeling and Analysis Framework (EMAF)
                 effort is to build on top of Ptolemy II and adapt Ptolemy II for the
                 rapid construction and configuration of modeling and analysis systems
                 that incorporate disparate technologies. The purpose of this
                 gap-filling project is to develop technologies for future
                 incorporation into large-scale modeling and analysis systems, with
                 specific focuses on scalable algorithm description, composition of
                 heterogeneous components, and synthesis of efficient deployable
                 decision-support systems that exploit multicore and distributed
                 computing platforms. In particular, we have applied the code
                 generation infrastructure developed under this MURI to a very large
                 problem consisting of roughly 13000 actors. We were able to reduce
                 the run time from roughly 10 minutes to 3 seconds.

                 \emph{US Army Research Laboratory}
                 
                 \noindent \emph{Scalable Composition of Subsystems (SCOS)}

                 Chris Winslow, winslow@arl.army.mil

                 The Scalable Composition of Subsystems (SCOS) project, was
                 funded by the Army Research Office in connection with the OSD
                 Software-Intensive Systems Producibility Initiative. The mission was
                 to research scalable techniques in software engineering based upon the
                 concepts inherent in modelbased composition. The overarching goal was
                 to show that these techniques will result in predictable and
                 understandable behaviors in Systems-of-Systems (SoS) environments. The
                 focus was on interaction between components (rather than the
                 conventional focus on transformation of data), and on composition,
                 which in this domain needs to be intrinsically concurrent (rather than
                 the conventional thread-based applique of concurrency on imperative
                 models).

                 \noindent \emph{Disciplined Design of Systems of Systems (DDoSoS)}

                 Chris Winslow, winslow@arl.army.mil

                 The Disciplined Design of Systems of Systems (ddosos) is sponsored by
                 the Army Research Laboratory (ARL) The project covers these areas:
                 Multiform Models of Time, Temporal Isolation, Hybrid Models, Correct
                 Composition, Linking Behaviors to Implementation and Design Drivers.

                 \emph{Lawrence Berkeley National Laboratory}

                 Michael Wetter, MWetter@lbl.gov

                 Researchers at the Lawrence Berkeley National Laboratory have developed the BCVTB:
                 ``The Building Controls Virtual Test Bed is a software
                 environment that allows expert users to couple
                 different simulation programs for co-simulation, and to
                 couple simulation programs with actual hardware. For
                 example, the BCVTB allows to simulate a building in
                 EnergyPlus and the HVAC and control system in Modelica,
                 while exchanging data between the software as they
                 simulate. The BCVTB is based on the Ptolemy II software
                 environment. The BCVTB allows expert users of
                 simulation to expand the capabilities of individual
                 programs by linking them to other programs. Due to the
                 different programs that may be involved in distributed
                 simulation, familiarity with configuring programs is
                 essential.''

                 \emph{Various Universities}
                 Kepler: A System for Scientific Workflows, is a cross-project collaboration to develop open source tools for Scientific Workflow Management and is currently based on the Ptolemy II system for heterogeneous concurrent modeling and design. 

                 The Kieler Project at the University of Kiel is ``a research project about enhancing the graphical model-based design of complex systems. The basic idea is to consistently employ automatic layout to all graphical components of the diagrams within the modeling environment. This opens up new possibilities for diagram creation and editing on the one hand and also new methods for dynamic visualizations of e.g. simulation runs on the other hand. Hence the focus of this project is the pragmatics of model-based system design, which can improve comprehensibility of diagrams, shorten development and change actions, and improve the analysis of dynamic behaviour.'' 

                \emph{Lockheed Martin Advanced Technology Laboratory}

                Trip Denton (ldenton@atl.lmco.com) 
                3 Executive Campus, 6th Floor; Cherry Hill, NJ, 08002, USA
                Work: 856-792-9071 Fax: 856-792-9925

                NAOMI Project (http://chess.eecs.berkeley.edu/naomi) (Also participating are Vanderbilt and UIUC) The purpose of the NAOMI project is to allow disparate modeling tools to be used to ether by tracking model changes within each system where a particular tool owns attributes of the overall design and provides attribute changes to other tools. The NAOMI project may result in useful technology that will allow easier collaboration on this MURI project. This project is using pedes-trian/automobile traffic lights as a design driver. We have integrated Ptolemy II to the Naomi frame-work, which allows different tools to own attributes and update other tools when changes occur to those attributes. We have transferred models that use graph transformation and event relationship graphs.

                


\subsubsection{Bruce Krogh}

\emph{National Science Foundation}

Helen Gill (hgill@nsf.gov)

Contributed to the development of the NSF Solicitation for Cyber-Physical Systems.

\emph{Lockheed Martin Advance Development Projects (ADP)}

Peter Stanfill (peter.o.stanfill@lmco.com)

Consultant to the LM team in the AFRL MCAR program.

