\subsection{Technology Assists, Transitions, and Transfers.}

\subsubsection{Vanderbilt}

\emph{Model-Integrated Computing (MIC) Tool Suite}

\begin{itemize}
\item Key components of Vanderbilt's MIC tool suite had several major releases during the course of the MURI project.  The ESMoL tool suite is published as an open source design environment and as part of the MIC distribution.
\item DARPA META project is part of DARPA's AVM program suite focusing on model-based design and verification of cyber-electro-mechanical systems.
After the first phase of META concluded at the end September 2011, DARPA selected Vanderbilt to lead a team  to develop and integrate  an open source META-X tool chain. The team includes PARC, SRI, MIT, Georgia Institute of Technology, SIFT/University of Oxford and Oregon State. Besides ISIS’ model-based design tools, the emerging META-X tool suite will incorporate cutting edge  tools and methods available from the research team as well as from the broad research community.  Results of this MURI have been fully transitioned into the software design part of META-X and constitute a crucial part of the overall design flow. 
\item Vanderbilt continued work with GM, Raytheon, Lockheed Martin, Boeing, and BAE Systems research groups on transitioning model-based design technologies into programs.
\item Vanderbilt applied the MIC tools with Boeing's FCS program to achieve precise architecture modeling and systems integration.
\end{itemize}

\emph{ESMoL Tools}

\begin{itemize}
\item The ESMoL tool suite developed as part of this program has been publicly released via the project Wiki site.  The most recent release was in November of 2011:

https://wiki.isis.vanderbilt.edu/hcddes/index.php/The\_ESMoL\_Tool

\item The ESMoL tools were integrated into the META tool suite for use in designing, integrating, simulating, and analyzing controller designs and their software realizations.
\end{itemize}

\subsubsection{UC Berkeley}

\emph{Ptolemy II 8.0}

                 Ptolemy II 8.0.1 \cite{PtolemyII801} shipped on October, 28, 2010.  New work in Ptolemy II 8.0.1 includes:
                 \begin{itemize}
                 \item Model Transformation - a framework for the analysis and transformation of actor models using model transformation techniques 
                 \item Ptera (Ptolemy Event Relationship Actor) Domain 
                 \item Causality Analysis: Updates to our non-conservative causality analysis for modal models within discrete-event (DE) systems 
                 \item Continuous and Modal Domains: a substantial rework of modal models and the underlying finite state machine infrastructure to make them work predictably and consistently across domains. 
                 \end{itemize}

\noindent \emph{Strategic Directions in Software At Scale},

                 In August 2010, the Office of Information Systems and Cyber Security (ISCS) within the Office of the Director, Defense Research and Engineering (DDR\&E) sponsored the Strategic Directions in Software at Scale (SaS) Workshop \cite{MayLeeJones11_StrategicDirectionsInSoftwareAtScale}. The SaS Workshop was hosted by the University of California, Berkeley. The goals of the workshop were to:
                 
                 \begin{itemize}
                 \item Identify new ideas and promising research directions in software engineering and computer science achievable in the short-, mid-, and long-term.
                 \item Identify opportunities for collaboration and engage in rich intellectual exchange of technical ideas.
                 \item Create a foundation for developing a DoD roadmap for SaS.
                 \item Begin to build a case for increasing DoD investment in software engineering and computer science research to strengthen the DoD’s software technology base.

                   Fifteen invited speakers gave presentations in the areas of software synthesis, robust and continuous behavior, temporal semantics, scalable composition, and software engineering process and methodology. Each speaker advocated a particular technical approach that could be the basis for a “Strategic Direction” in future software research. To capture the quality and promise of the technical approaches, attendees were asked to rate each presentation with respect to six evaluation criteria.
                   

                   The overall best technical approaches, as assessed by the attendees, were "Temporal Semantics in Concurrent and Distributed Software"—Edward Lee, "Is Distributed Consistency Scalable?"—Ken Birman, and "The Effect of Software (and Communication) Reliability and Security on Control Systems"—Bruno Sinopoli.
                 \end{itemize}

\noindent \emph{Ptolemy Miniconference}

                 The Ninth Biennial Ptolemy Miniconference was held on February 16, 2011.  We had 70 attendees from around the world.

                 The Ptolemy Miniconference is an opportunity for research
                 collaborators and Ptolemy users and extenders from industry, academia,
                 and government to get together, present their work to the Ptolemy
                 community, and hear about related research and results.

                 \begin{itemize}
                 \item Edward A. Lee (Berkeley). "Ptolemy
                   Miniconferences." \cite{Lee11_PtolemyMiniconferences} 
                 \item Jianwu Wang, Daniel Crawl, Ilkay Altintas, Chad Berkley, Matt
                   Jones (San Diego Supercomputer Center and UC Santa Barbara). "Distributed Execution Architectures in Kepler."
                   \cite{WangCrawlAltintasBerkleyJones11_DistributedExecutionArchitecturesInKepler}
                 \item Patricia Derler, Jia Zou, Slobodan Matic, John Eidson (Berkeley). "Modeling
                   Distributed Real-Time Systems with Ptolemy II."
                   \cite{DerlerZouMaticEidson11_ModelingDistributedRealTimeSystemsWithPtolemyII}
                 \item Stavros Tripakis, Edward A. Lee (Berkeley). "Semantics of Modal Models in
                   Ptolemy."
                   \cite{TripakisLee11_SemanticsOfModalModelsInPtolemy}
                 \item Charles Shelton, Elizabeth Latronico, Ben Lickly (Bosch, Berkeley). "Static
                   Analysis using the Ptolemy II Ontologies Package."
                   \cite{SheltonLatronicoLickly11_StaticAnalysisUsingPtolemyIIOntologiesPackage}
                 \item Jan Reineke, Isaac Liu, Gage W. Eads, Stephen A. Edwards,
                   Sungjun Kim, Hiren D. Patel (Berkeley, Columbia, Waterloo). "To Meet or Not to Meet the Deadline."
                   \cite{ReinekeLiuEadsEdwardsKimPatel11_ToMeetOrNotToMeetDeadline}
                 \item Shuvra S. Bhattacharyya (University of Maryland). "The Dataflow Interchange Format:
                   Towards Co-Design of DSP-oriented Dataflow Models and
                   Transformations." 
                   \cite{Bhattacharyya11_DataflowInterchangeFormatTowardsCoDesignOfDSPoriented}
                 \item Sven Koehler, Bertram Ludaescher, Timothy McPhillips (UC Davis). "Workflow Recovery for Different Models of Computation and Models of Provenance." 
                   \cite{KoehlerLudaescherMcPhillips11_WorkflowRecoveryForDifferentModelsOfComputationModels}
                 \item Kaushik Ravindran, Murali Parthasarathy (National Instruments). "Design, Analysis, and
                   Implementation of Static Dataflow Models for Hardware Targets."
                   \cite{RavindranParthasarathy11_DesignAnalysisImplementationOfStaticDataflowModels}
                 \item Gongjing Cao, Lei Dou, Quinn Hart, Bertram
                   Ludaescher (UC Davis). "Kepler/G-Pack: A Kepler Package Using the Google Cloud
                   for Interactive Scientific Workflows."
                   \cite{CaoDouHartLudaescher11_KeplerGPackKeplerPackageUsingGoogleCloudForInteractive}
                 \item Anne Ngu, George Chin Jr. (Texas State Univ., Pacific NW National Lab). "Context Aware Actors." 
                   \cite{NguChinJr11_ContextAwareActors}
                 \item Dai Nguyen Bui, Stavros Tripakis, Marc Geilen, Bert Rodiers,
                   Edward A. Lee (Berkeley). "Modular Code Generation."
                   \cite{BuiTripakisGeilenRodiersLee11_ModularCodeGeneration}
                 \item Edward A. Lee (Berkeley). "The Ptolemy Project: Advancing System Design." 
                   \cite{Lee11_PtolemyProjectAdvancingSystemDesign}
                 \end{itemize}

                 In addition, the following posters were presented, see \\
                 \href{http://ptolemy.eecs.berkeley.edu/conferences/11/presentations.htm}{http://ptolemy.eecs.berkeley.edu/conferences/11/presentations.htm}
                 \begin{itemize}
                 \item Gage Eads (Berkeley), ``Deadline Instructions in a PRET Architecture.''
                 \item Shanna-Shaye Forbes (Berkeley), ``Error Handling in Model-Based
                   design for Real-Time Systems.''
                 \item Soheil Ghiasi, Matin Hashemi (UC Davis), ``Malleable Dataflow Specification: An Essential Ingredient for Resource-Scalable Implementations.''
                 \item Isaac Liu, Jan Reineke (Berkeley), ``A PRET Architecture Supporting Concurrent Programs with Composable Timing Properties.''
                 \item Slobodan Matic, Ilge Akkaya, and John Eidson (Berkeley), ``The Distributed Power System Test Case for Distributed Real-Time Systems.''
                 \item Christian Motika, Hauke Fuhrmann, Miro Spönemann Reinhard von
                   Hanxleden (Kiel University), ``KIELER Actor Oriented Modeling.''
                 \item Darryl Koivisto, Deepak Shankar (Mirabilis), ``Using Ptolemy/VisualSim for Internet-based model Sharing and Communication.''
                 \item Chris Shaver (Berkeley), ``Alternative Syntactic Representations of Graph-Based Models.''
                 \item Chris Shaver, Dai Bui, Stavros Tripakis(Berkeley), ``Multidimensional Dataflow Models.''
                 \item Elizabeth Latronico, Charles Shelton, Ben Lickly (Bosch, Berkeley), ``Lattice Composition for Ontology Analysis.''
                 \item Ben Lickly, Charles Shelton, Elizabeth Latronico (Bosch, Berkeley), ``Practical Ontologies with Infinite Lattices.''
                 \item Andreas Thuy (University of Paderborn), `` 	Towards flexible and robust cyber-physical-systems through self organization.''
                 \item Stavros Tripakis, Marc Geilen, Maarten Wiggers (Berkeley, TU
                   Eindhoven), ``The Earlier the Better: A Theory of Timed Actor Interfaces.''
                 \item Mike Wirthlin (Brigham Young University), ``Automated Bit-Width Analysis Using Ptolemy.''
                 \item Michael Zimmer (Berkeley), ``IEEE 1588 Time Synchronization for Real-Time Distributed Systems.''
                 \item Jia Zou, Slobodan Matic, John Eidson (Berkeley), ``From PTIDES
                   to PtidyOS: Programming Distributed Real-Time Embedded Systems.''
                 \end{itemize}



