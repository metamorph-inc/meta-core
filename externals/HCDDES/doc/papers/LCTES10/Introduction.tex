\section{Introduction}

High confidence embedded systems development features multi-faceted design activities
complemented by model and data analysis to ensure the correct operation of the final
product.  Such analysis is multi-faceted due to the heterogeneity of domains 
involved in the problems (e.g. cyber-physical systems), leading to high costs and
extended schedules for development efforts.  Analyses suffer from difficulties
of scale due to state and parameter space explosion problems.  Further, interactions
between design domains (such as the physical continous-time and computational 
discrete-event) are poorly understood, requiring analyses to be overly conservative.
Recently the research community for embedded systems has explored 
constructive techniques for design and analysis, where system correctness properties
may be inferred from component correctness properties and more abstract interaction 
models defined on the interconnection structure of the design.  Constructive methods
strive to dramatically reduce state explosion problems by assuring correctness during
model construction, while avoiding overly conservative approximations.

BIP \cite{verif:bbs06} is a framework for the constructive assurance of computational 
properties in embedded systems design.  BIP is a component framework for constructing 
systems by superimposing three layers of modelling: Behaviour, Interaction, and Priority. 
The first layer consists of a set of atomic components represented by transition systems.  
The second layer models Interaction between components. Interactions are sets of ports 
specified by connectors \cite{verif:bs08,verif:bs09}. Priority, given by a strict partial 
order on interactions, is used to enforce scheduling policies applied to interactions of 
the second layer. The BIP component framework has a formal operational semantics given in 
terms of Labelled Transition Systems and Structural Operational Semantics derivation rules.

The BIP language offers primitives and constructs for modelling and composing complex 
behaviour from atomic components.  Atomic components are communicating automata extended 
with C functions and data. Transitions are labelled with sets of communication ports. 
Compound components are obtained from subcomponents by specifying connectors and 
priorities.

ESMoL is another modeling language which aims to experiment with encoding constructive 
techniques for high-confidence systems into a single modeling environment.  Prior work has 
dealt with modeling and deployment for distributed control systems implemented using an 
implementation of the time-triggered model of computation\cite{timed:frodo, modeling:aces08, sched:analysis}.  
While the time-triggered architecture \cite{timed:tta2} provides many guarantees for
deterministic execution and fault tolerance, extensions are necessary to handle 
sporadic traffic.  Some hardware solutions aim to address these problems, such as 
time-triggered ethernet\cite{timed:ttethernet} and AFDX\cite{afdxspec} (to name a few of 
the most popular).

The execution environment addresses only part of the problem.  High-confidence designs
require assurances of correct execution, including guarantees for dynamic stability, 
schedulability and deadlock freedom.  The ESMoL development effort seeks to address such problems 
constructively.  In order to provide assurances of deadlock freedom, we aim to integrate 
asynchronous designs modeled in BIP into our modeling, analysis, and deployment environment.

The integration of BIP and ESMoL seeks to address a number of outstanding problems, and
confers benefits on both tools:

\begin{enumerate}
 \item ESMoL offers platform and deployment modeling, along with wrapper code generation
for distributed messaging and real-time execution.  Using ESMoL, BIP models can easily be 
tested against multiple deployment configurations.
 \item As mentioned, BIP offers constructive deadlock analysis, which is essential to 
support high-confidence design activities.  ESMoL resimulation could include BIP models 
for communication protocols and task schedulers, leading to a more accurate determination
of possible deadlocks.
 \item ESMoL currently lacks asynchronous semantics in both the modeling environment and
runtime, so extensions to support BIP models will address this deficiency.
\end{enumerate}

In this report we describe the first phase of the integration effort: importing BIP models
into ESMoL, and creating platform-specific simulations.  To achieve this, a number of 
technical challenges must be overcome:

\begin{enumerate}
 \item Creating an appropriate syntactic transformation from BIP models to ESMoL models.
 \item Mapping BIP component interaction models to appropriate message transfer models
for deployment, while preserving semantics.
 \item Extending the ESMoL code generation framework to support imported BIP models.
 \item Extending the FRODO runtime (used in the ESMoL tools) to support asynchronous
operation.
\end{enumerate}

We will cover design details for each of these challenges by describing the 
flow of an example model through the tools.  We will also discuss current development 
progress for each part, and include results where appropriate.  Section \ref{section:solution}
introduces the example and describes our solution in more detail.  Section \ref{section:trans} 
covers the syntactic transformation between BIP and ESMoL.  Section \ref{section:semantics}
describes the semantic transformation from generic BIP connectors to send-receive messaging
required to distribute BIP execution on a network.  Execution environment extensions and 
simulation results are discussed in Section \ref{section:exec}.  Finally we visit
related work (Sec. \ref{section:related}), future work, and conclusions (Sec. \ref{section:conc}).