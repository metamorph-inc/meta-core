% RTAS Demo Abstract
%
\ifx\techreport\undefined
\documentclass[10pt, conference, compsocconf]{IEEEtran}
\else
\documentclass[times,10pt]{article}
\fi
%\IEEEoverridecommandlockouts
%\documentclass{acm_proc_article-sp}
%NOTE for dissertation diable noinfo until ready to submit
%nesc additions
%\documentclass{svmult}
%\usepackage{multicol}
%nesc additions
\usepackage{enumerate}
\usepackage{times}
\usepackage{latexsym}
\usepackage{xspace}
\usepackage{amssymb,amsmath}
\interdisplaylinepenalty=2500
\usepackage{cite}
\ifx\pdfoutput\undefined
  \message{NOT IN PDF MODE}
  \usepackage{graphicx}
\else
  \message{UNCOMMENT if really IN PDF MODE}
  \usepackage{graphicx}
  %\usepackage[pdftex]{graphicx}
\fi

\ifx\pdfoutput\undefined
  \message{NOT IN PDF MODE}
  \usepackage[hypertex]{hyperref}
\else
  \message{UNCOMMENT if really IN PDF MODE}
  \usepackage[hypertex]{hyperref}
  %\usepackage[pdftex,hypertexnames=false]{hyperref}
\fi

%\usepackage{html}
\usepackage{url}
\usepackage{multirow}
\usepackage{subfigure}

%\usepackage{setspace}
%\newtheorem{proof}{Proof}
\newtheorem{remark}{Remark}
\newtheorem{theorem}{Theorem}
\newtheorem{lemma}{Lemma}
\newtheorem{proposition}{Proposition}
\newtheorem{definition}{Definition}
\newtheorem{assumption}{Assumption}
\newtheorem{notation}{Notation}
\newtheorem{corollary}{Corollary}
%\setlength {\topmargin} {0 mm}
%\setlength {\headsep} {0 mm}
%\setlength {\headheight} {0 in}
%\setlength {\voffset} {0 mm}
%\setlength {\oddsidemargin} {0 mm}
%\setlength {\evensidemargin} {0 mm}
%\setlength {\hoffset} {0 mm}
%\setlength {\textwidth} {6.5 in}
%\setlength {\textheight} {9 in}
\newcommand{\MOAP}[0]{MOAP}
\newcommand{\RAGOBOT}[0]{RAGOBOT}
\newcommand{\Deluge}[0]{Deluge}
\newcommand{\modules}[0]{{\it modules}}
\newcommand{\networkingstderr}[0]{{\it networking standard error}}
\newcommand{\networkingconfiguration}[0]{{\it networking configuration}}
\newcommand{\controlstdin}[0]{{\it control standard input}}
\newcommand{\controlstdio}[0]{{\it control standard input output}}
\newcommand{\controlstdout}[0]{{\it control standard output}}
\newcommand{\buildutilities}[0]{{\it build utilities}}
\newcommand{\sosutilities}[0]{{\it sos utilities}}
\newcommand{\freematmodpid}[0]{{\it FREEMAT\_MOD\_PID}}
\newcommand{\sossrv}[0]{sossrv}
\newcommand{\neclabsimrun}[0]{{\it \neclab\_\{sim,run\}}}
\newcommand{\neclabsim}[0]{{\it \neclab\_sim}}
\newcommand{\neclabrun}[0]{{\it \neclab\_run}}
\newcommand{\GNU}[0]{GNU}
\newcommand{\SOS}[0]{{\it SOS}}
\newcommand{\GPS}[0]{GPS}
\newcommand{\gps}[0]{gps}
\newcommand{\loc}[0]{loc}
\newcommand{\XYZ}[0]{XYZ}
\newcommand{\LOC}[0]{LOC}
\newcommand{\XML}[0]{XML}
\newcommand{\moduled}[0]{moduled}
\newcommand{\ps}[0]{ps}
\newcommand{\proto}[0]{proto}
\newcommand{\pc}[0]{pc}
\newcommand{\pid}[0]{pid}
\newcommand{\fifo}[0]{fifo}
\newcommand{\makerules}[0]{makerules}
\newcommand{\unix}[0]{unix}
\newcommand{\Yale}[0]{Yale}
\newcommand{\BLAS}[0]{BLAS}
\newcommand{\ATLAS}[0]{ATLAS}
\newcommand{\perl}[0]{perl}
\newcommand{\avr}[0]{avr}
\newcommand{\tmp}[0]{tmp}
\newcommand{\src}[0]{src}
\newcommand{\gcc}[0]{gcc}
\newcommand{\GCC}[0]{GCC}
\newcommand{\binutils}[0]{binutils}
\newcommand{\libc}[0]{libc}
\newcommand{\uisp}[0]{uisp}
\newcommand{\tcl}[0]{tcl}
\newcommand{\tk}[0]{tk}
\newcommand{\python}[0]{python}
\newcommand{\swig}[0]{SWIG}
\newcommand{\pexpect}[0]{pexpect}
\newcommand{\neclab}[0]{{\it neclab}}
\newcommand{\CVShead}[0]{CVShead}
\newcommand{\MICAbot}[0]{MICAbot}
\newcommand{\moteLab}[0]{moteLab}
\newcommand{\necroot}[0]{{\it necroot}}
\newcommand{\localhost}[0]{HOST\_PC}
\newcommand{\wnecs}[0]{{\it wnecs}}
\newcommand{\Wnecs}[0]{{\it Wnecs}}
\newcommand{\necs}[0]{{\it necs}}
\newcommand{\necssh}[0]{necssh}
\newcommand{\Necssh}[0]{Necssh}
\newcommand{\MATLAB}[0]{MATLAB}
\newcommand{\RTLinux}[0]{RTLinux}
\newcommand{\Simulink}[0]{Simulink}
\newcommand{\TrueTime}[0]{TrueTime}
\newcommand{\FreeMat}[0]{FreeMat}
\newcommand{\freematclient}[0]{{\it \FreeMat\ client}}
\newcommand{\freematutilities}[0]{{\it \FreeMat\ utilities}}
\newcommand{\API}[0]{API}
\newcommand{\NESL}[0]{NESL}
\newcommand{\Kepner}[0]{Kepner}
\newcommand{\MPI}[0]{MPI}
\newcommand{\TinyOS}[0]{{\it TinyOS}}
\newcommand{\SOSROOT}[0]{SOSROOT}
\newcommand{\NECLABROOT}[0]{NECLABROOT}
\newcommand{\MatlabMPI}[0]{MatlabMPI}
\newcommand{\Scilab}[0]{Scilab}
\newcommand{\Wincon}[0]{Wincon}
\newcommand{\DEC}[0]{DEC}
\newcommand{\networkingsource}[0]{{\it networking source}}
\newcommand{\networkingsink}[0]{{\it networking sink}}
\newcommand{\networkingfilter}[0]{{\it networking filter}}
\newcommand{\networkingmessagepipe}[0]{{\it networking message pipe}}
\newcommand{\networkingmessagetype}[0]{{\it networking message type}}
\newcommand{\controlmessagetype}[0]{{\it control message type}}
\newcommand{\Intel}[0]{Intel}
\newcommand{\Xerox}[0]{Xerox}
\newcommand{\OpenBSD}[0]{OpenBSD}
\newcommand{\NetBSD}[0]{NetBSD}
\newcommand{\FreeBSD}[0]{FreeBSD}
\newcommand{\Linux}[0]{Linux}
\newcommand{\MICAt}[0]{MICA2}
\newcommand{\MICAz}[0]{MICAz}
\newcommand{\MICAtz}[0]{\MICAt\ and \MICAz}
\newcommand{\MIB}[0]{MIB510}
\newcommand{\Crossbow}[0]{Crossbow}
\newcommand{\MultiQ}[0]{MultiQ}
\newcommand{\necsshdaemon}[0]{necssh-daemon}
\newcommand{\MNgateway}[0]{MICA2\_N\_gateway}
\newcommand{\MNsensor}[0]{MICA2\_N\_sensor}
\newcommand{\MNcontrollerA}[0]{MICA2\_N\_controller-A}
\newcommand{\MNcontrollerB}[0]{MICA2\_N\_controller-B}
\newcommand{\MNactuator}[0]{MICA2\_N\_actuator}
\newcommand{\modd}[0]{modd}
\newcommand{\gw}[0]{gw}
\newcommand{\nconf}[0]{conf}
\newcommand{\cmd}[0]{cmd}
\newcommand{\dev}[0]{dev}
\newcommand{\ttyS}[0]{ttyS}
\newcommand{\serialport}[0]{/\dev/\ttyS0}
\newcommand{\Atmel}[0]{Atmel}
\newcommand{\ATmega}[0]{ATmega}
\newcommand{\sosbase}[0]{sosbase}
\newcommand{\sosbasepid}[0]{SOSBASE\_PID}
%new macros for mtt_gov.tex
\newcommand{\TWS}[0]{{\it TWS}}
\newcommand{\MTT}[0]{{\it MTT}}
\newcommand{\MHT}[0]{{\it MHT}}
\newcommand{\Kalman}[0]{Kalman}
\newcommand{\Nyquist}[0]{Nyquist}
\newcommand{\JPDA}[0]{{\it JPDA}}
\newcommand{\PDA}[0]{{\it PDA}}
\newcommand{\NN}[0]{{\it NN}}
\newcommand{\NLP}[0]{{\it NLP}}
\newcommand{\MLAB}[0]{{\it MLAB}}
\newcommand{\MCMCDA}[0]{{\it MCMCDA}}
\newcommand{\JMPD}[0]{{\it JMPD}}
\newcommand{\wsn}[0]{{\it wsn}}
\newcommand{\MJLS}[0]{{\it MJLS}}
\newcommand{\MSS}[0]{{\it MSS}}
\newcommand{\EMSS}[0]{{\it EMSS}}
\newcommand{\SSa}[0]{{\it SS}}
\newcommand{\LMI}[0]{{\it LMI}}
\newcommand{\UGAS}[0]{{\it UGAS}}
\newcommand{\UGES}[0]{{\it UGES}}
\newcommand{\TOD}[0]{{\it TOD}}
\newcommand{\MATI}[0]{{\it MATI}}
\newcommand{\ISS}[0]{{\it ISS}}
\newcommand{\TCP}[0]{{\it TCP}}
\newcommand{\TDMA}[0]{{\it TDMA}}
\newcommand{\ALOHA}[0]{{\it ALOHA}}
\newcommand{\MAC}[0]{{\it MAC}}
\newcommand{\CSMA}[0]{{\it CSMA}}
\newcommand{\MIMO}[0]{{\it MIMO}}
\newcommand{\nstwo}[0]{{\it NS-2}}
\newcommand{\Lyapunov}[0]{Lyapunov}
\newcommand{\LTI}[0]{{\it LTI}}
\newcommand{\UAV}[0]{{\it UAV}}
\newcommand{\SA}[0]{{\it SA}}
\newcommand{\SLAT}[0]{{\it SLAT}}
\newcommand{\LaSLAT}[0]{{\it LaSLAT}}
\newcommand{\EKF}[0]{{\it EKF}}
\newcommand{\MCMC}[0]{{\it MCMC}}
\newcommand{\MBP}[0]{{\it MBP}}
\newcommand{\MAP}[0]{{\it MAP}}
\def\pus{passive upsampler}
\def\pds{passive downsampler}
\def\PATRU{PATRU}
\def\PDS{PDS}
\def\PUS{PUS}
\def\CSO{{\it CSO}}
\def\SPR{{\it SPR}}
\def\CSS{{\it CSS}}
\def\EPA{{\it EPA}}
\def\ILS{{\it ILS}}
\def\BIBO{{\it BIBO}}
\def\antistable{{\it anti-stable}}
\def\wavevars{{\it wave variables}}
\def\wavevar{{\it wave variable}}
\def\Wavevars{{\it Wave variables}}
\def\semistable{{\it semi-stable}}
\def\asm{{\it asymptotic stability in the} $m^{th}$ {\it mean}}
\def\telemanipulation{telemanipulation}
\def\Telemanipulation{Telemanipulation}
\def\TELEMANIPULATION{TELEMANIPULATION}
\def\sop{{\it strictly-output passive}}
\def\sopy{{\it strictly-output passivity}}
\def\spr{{\it strictly-positive real}}
\def\Passive{{\it Passive}}
\def\passive{{\it passive}}
\def\passivity{{\it passivity}}
\def\sip{{\it strictly-input passive}}
\def\sipy{{\it strictly-input passivity}}
\def\paa{{\it parameter adaptation algorithms}}
\def\adaptivecontrol{{\it adaptive control}}
\def\els{{\it Euler-Lagrange Systems}}
\def\IPESH{{\it IPESH}}
\def\IPES{{\it IPES}}
\def\ipes{{\it inner-product equivalent sampler}}
\def\ipesh{{\it inner-product equivalent sample and hold}}
\def\ipeusds{{\it inner-product equivalent up-sampler/ down-sampler}}
\def\IPEUD{IPEUD}
\def\PS{PS}
\def\PH{PH}
\def\LHS{LHS}
\def\RHS{RHS}
\def\PCH{{\it PCH}}
\def\ZOH{{\it ZOH}}
\def\CT{{\it CT}}
\def\DT{{\it DT}}
\def\Hamiltonian{{\it Hamiltonian}}
\def\Schur{Schur}
\def\supremum{supremum}
%Math macros
\def\bydefn{\stackrel{\triangle}{=}}
\def\iff{iff}
\DeclareMathOperator{\diag}{diag}
\newcommand{\argmax}[1]{\hbox{\space \raise-2mm\hbox{$\max \atop #1$}
    \space}}
\newcommand{\argmin}[1]{\hbox{\space \raise-2mm\hbox{$\min \atop #1$}
    \space}}
\newcommand{\mlim}[2]{#1 \atop {\lim \atop #2}}
\newcommand{\msup}[1]{\hbox{\space \raise-2mm\hbox{$\sup \atop #1$}
    \space}}
\newcommand{\innerprod}[2]{\langle #1,#2 \rangle}
\newcommand{\eqn}[1]{(\ref{#1})}
\newcommand{\norm}[2]{\|#1\|_{#2}}
\newcommand{\twonorm}[1]{\norm{#1}{2}}
\def\tr{^{\mathsf{T}}}
\newcommand{\xovec}[1]{\begin{bmatrix}\hat{x}(#1)\\x(#1)\end{bmatrix}}
\newcommand{\sgn}[1]{\mathtt{sgn}(#1)}


\begin{document}

\title{Demo Abstract: The ESMoL Modeling Language and Tools for Synthesizing and Simulating Real-Time Embedded Systems}

\author{\IEEEauthorblockN{Joseph Porter, Zsolt Lattmann, Graham Hemingway,   
Nagabhushan Mahadevan, Sandeep Neema, \\
Harmon Nine, Nicholas Kottenstette, P\'{e}ter V\"{o}lgyesi, G\'{a}bor Karsai, and J\'{a}nos Sztipanovits}
\IEEEauthorblockA{Institute for Software Integrated Systems \\
Vanderbilt University\\
Nashville, TN 37235, USA \\
Email: jporter@isis.vanderbilt.edu\\
Telephone: (615) 343--4641}
}

\maketitle

\begin{abstract}
High-confidence embedded real-time designs stretch the demands placed
on design and development tools.  We will demonstrate the design and
testing of an embedded control system built using the ESMoL modeling
language and supporting tools.  ESMoL adds distributed deployment 
concepts to Simulink designs, and integrates scheduling analysis as well
as platform-specific simulation.  The testing system includes a simulated
physical plant running in a hardware-in-the-loop configuration with the
actual embedded controller.
\end{abstract}

\section{Introduction}

Many classes of real-time embedded systems require assurance that 
engineering designs and implementations are safe, secure, and correct.  
Examples include software for medical devices, weapons control systems, 
and aircraft flight control.  The time and effort (and consequently cost)
of such assurance stands in contrast with the cost and schedule pressures
of production engineering.

The ESMoL (Embedded Systems Modeling Language) domain-specific modeling 
language (DSML) provides user modeling constructs for designing 
embedded systems -- including functional specifications in Simulink, 
models for distributed computing platforms, and deployment models assigning 
functions to tasks on computing nodes\cite{aces08}.  The language structure 
and the supporting Model-Integrated Computing (MIC) tools \cite{mic:metaprog} 
enable the integration of analysis tools to provide design assurance for use 
in high-confidence designs.

\begin{figure}[!b]
\centering
\includegraphics[width=0.9\columnwidth]{vdiagram}
\caption{Embedded development workflow supported by the tools.}
\label{fig:vdiagram}
\end{figure}

The development workflow supported by the tools is depicted in 
Fig. \ref{fig:vdiagram}.  After the Requirements Specification 
has been created, designers start with familiar tools like Simulink to
design and simulate the functional control system (Control Design).  
The Simulink design is imported into the modeling 
environment, and then annotated with Software Architecture and Component 
Design concepts.  Platform models (Hardware and System Architecture 
Design) represent the hardware architecture.  Componentized functions 
and signals are then assigned to tasks and messages deployed on the 
hardware (Software Deployment). From the composed model we can perform 
static real-time schedule determination and synthesize code for a distributed 
platform -- assuming the platform provides clocked synchronous task execution 
and time-triggered messaging services.  

The tools currently support development for unmanned aerial vehicles,
in particular the Starmac quadrotor helicopter \cite{HRWDJT04}.  The control
architecture is based on a fast inner-loop controller for attitude (here simplified
to angle) and a slower outer-loop controller for position (here simplified to a
scalar value).  Our control technique is based on a composition of nested passive 
(PD) controllers to stabilize a process modeled as a cascade of continuous-time 
systems \cite[Fig.~1]{kottenstette08:_digit_passiv_attit_and_altit}.  

\section{Tool Features and Progress}

The enabling ideas (beyond the existing capabilities of model-based 
development) are the integration of analysis and simulation tools to 
provide meaningful feedback to designers as early as possible in the 
development cycle, and a search for decoupling techniques which help 
make required analyses tractible.  This demonstration will focus on 
the first aspect of our work (integrated analysis and simulation early in the
design cycle) for embedded software synthesized from models.  A detailed presentation 
of the design philosophy, intended uses and features, and limitations 
of the language and its tools is available in \cite{aces08}.  A few 
features are noteworthy:

\begin{enumerate}
\item   The TrueTime
toolkit \cite{TrueTime} extends Simulink with blocks for modeling 
distributed platforms to simulate uncertainties due to distributed processing. 
Our modeling tools allow resimulation of the design using TrueTime
blocks synthesized from the deployment model.  A single design model targets both the 
continuous-time platform simulation environment of TrueTime and the tasks in the 
actual distributed system.  Platform-specific simulation exposes potentially
destabilizing effects of communication delays before physical system construction.
\item Platform communication delays are captured explicitly in the models.
An integrated scheduling tool uses these timing parameters for analysis.  
Computed schedules are fed back into the modeling environment for  
TrueTime simulations and for the final physical deployment.  Designers can 
iterate over possible component and plaform design alternatives and check 
schedulability.
\item In contrast to code generation tools like Real-Time workshop
\cite{mathworks:tools} -- which generate code for individual processing 
nodes -- ESMoL generates the task configuration code as well 
as the communication glue code for the
distributed platforms on which it runs.  The network may be
a heterogeneous collection of processors and buses, as long as processing
node and communication link types are supported by the code generators.
\item The entire tool infrastructure has been built around a
single abstract semantic model, greatly simplifying the process of 
integrating additional analysis and synthesis tools.  This ensures
that all integrated tools have a single, consistent view of the 
modeling language semantics.
\end{enumerate}

Current development includes extension of the TrueTime simulation generator, 
investigations into automated quantization of the digital controllers 
(subject to safety constraints), and consideration of other tools for verification 
of control properties and deadlocks.  In particular, integration of new
tools will require expansion of the language to include
specification of richer operational models.

\section{Tool Demonstration}

The demonstration will display the following aspects the ESMoL modeling 
language and tools:

\begin{enumerate}
\item Synthesis of a Simulink control design from the GME modeling 
environment \cite{mic:gme} to run on a simple distributed embedded processing
system (Gumstix processor with an attached Robostix I/O processor
\cite{gumstix}).
\item Computation of a cyclic real-time schedule using the integrated 
Gecode constraint solver\cite{gecode}\cite{sw:offlinescheduling}.
\item Execution of the synthesized control functions on a portable 
time-triggered virtual machine (FRODO\cite{RT_Thesis}) running on the 
embedded hardware.
\item Simulation of the control loops using a hardware-in-the-loop 
(HIL) simulator communicating with the embedded system through a physical 
interface.  The HIL simulator is a PC running the Mathworks' xPC target
\cite{mathworks:tools}.  Fig. \ref{fig:sim_arch} shows the structure of
the simulation system.
\end{enumerate}

\begin{figure}[!t]
\centering
\includegraphics[width=2.5in]{sim_arch}
\caption{Hardware-in-the-loop simulation architecture. The high-end
PXA processor runs generic Linux, and the low-end AVR microcontroller runs FreeRTOS.
The two communicate via an $I^2C$ bus using a lightweight time-triggered messaging
protocol. Communication to the plant simulator PC occurs over multiple 
high-speed serial lines.}
\label{fig:sim_arch}
\end{figure}

\section*{Acknowledgment}

This work is sponsored in part by the National Science Foundation 
(grant NSF-CCF-0820088) and by the Air Force Office of Scientific 
Research, USAF (grant/contract number FA9550-06-0312).  The views 
and conclusions contained herein are those of the authors and 
should not be interpreted as necessarily representing the official 
policies or endorsements, either expressed or implied, of the Air 
Force Office of Scientific Research or the U.S. Government.

\bibliographystyle{IEEEtran}
\bibliography{rtas_demo_2009}

\end{document}
