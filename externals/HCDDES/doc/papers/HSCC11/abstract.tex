\begin{abstract}

Time-triggered execution provides a theoretically strong basis for the design of safety-critical embedded systems.  Ideally, the design of such systems would not be limited to strictly time-triggered execution semantics.  Indeed, in many situations event-triggered semantics provide a more natural and intuitive approach for designers.  What is needed is a meaningful blending of time- and event-triggered execution semantics.  Any approach should maintain the analytic behavior and robust fault tolerance of time-triggering while providing the flexibility of event-triggering.  Obviously, any such solution will have trade-offs and compromises compared to purely time- or event-triggered execution.

%Any mention of existing approaches or their drawbacks?

In this paper we present an environment for modeling, analyzing, and executing systems with mixed time- and event -triggered semantics.  First, we present the ESMoL modeling language which provides an intuitive means of describing mixed execution-type systems.  Integrated into this environment is a tool for performing static schedulability analysis of time- and event-triggered systems.  Once a system has been designed and analyzed, the ESMoL toolchain can synthesize a complete implementation that realizes the system and its behaviors.


\end{abstract}


% A category with the (minimum) three required fields
\category{H.4}{Information Systems Applications}{Miscellaneous}
%A category including the fourth, optional field follows...
\category{D.2.8}{Software Engineering}{Metrics}[complexity measures, performance measures]
\terms{time-triggered, event-triggered}
%\keywords{ACM proceedings, \LaTeX, text tagging} % NOT required for Proceedings
