As this research effort is a work in progress, we conclude with a brief summary of
the next steps and future objectives for each of the tools presented.   We must 
keep in mind the final goal of verifiable and certifiable software for embedded 
systems.  This section contains forward-looking statements.

\SubSection{Software architecture specification}
The chief limitation of our software architecture tools is the one-way design flow 
from the SL/SF design, through componentization, down to the final 
code. We aim to improve the ability to send design information back to the earlier 
stages of the design as neeeded.  For example, platform-specific simulations may
indicate that jitter or quantization effects will impact the initial assumptions
of a control design.  Representing that data to control designers in a meaningful 
way will allow design changes without excessive workflow iterations.  
Schedulability is another area where downstream software 
design tools can provide meaningful feedback to the original design engineers.

\SubSection{Code generation}
The abstract model in the code generator opens the door for a number of
potential static analysis and verification opportunities.  The current toolchain 
includes two code generators that produce C (and Java) source code from 
(single-rate) subsystems in Simulink and Stateflow models. The code generators have 
been implemented using graph transformation techniques, and they produce an AST from 
which the actual code is printed. To assist in system-level or functional code
verification the AST could be extended to carry over information from the original 
model, thus providing guidance for the source code-level verification tool 
regarding the original model from which the code was generated and its properties.
We believe this can significantly improve the performance of the verification step 
because the verifier does not have to �reverse engineer� the high-level 
abstractions from the source code, as the abstractions are readily available in
the models. 

\SubSection{Scheduling}
We aim to expand the scheduling tools to include specific time-triggered models.
One simple example is that of adding constraints to support the requirements of
the TTTech TTP/C hardware.   Another avenue for research is the exploration of 
interactions between the resource allocation model (via schedules) with other 
system objectives which can be modeled by constraint or optimization problems in
other domains (such as continuous stability in the control design).  

\SubSection{Extending the modeling of the execution platform}
The formal DEVS model is a big step towards providing guaranteed safety and 
performance in time-triggered control system designs.  DEVS also supports pure
event-triggered behaviors in addtion to timed models.  Experimentation with this 
capability will hopefully lead to a better understanding of the limitations of 
heterogeneous component interactions in our system designs. 

Platform simulation also opens up opportunities for exploration.  The TrueTime tool
suite from Lund University \cite{TrueTime} extends Simulink models with 
concepts for modeling distributed platforms, scheduling policies, and communication 
protocols.   TrueTime promises to help characterize behavioral changes due to the
distribution of functionality over networked processors.

\SubSection{Extending the capabilities of the execution platform}
As the capabilities of the formal models expand, we aim to extend our portable
VM implementation to manage heterogeneous behaviors.  The VM will also be ported
to other operating platforms, including diverse hardware and RTOSes such as QNX 
and uC-OS.  Different platforms provide different levels of assurance regarding
timing, determinism, and resource management.  These differences will need to be
reflected in the models.  New features may also be added to the VM as required to 
support interaction idioms such as remote procedure calls or rendezvous.  We may 
also require additional component services such as health monitoring, fault management, 
robust clock synchronization, or failover.  

