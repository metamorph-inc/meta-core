The domain of choice for this research is that of distributed and embedded control
systems.  Accordingly, the formal MoC chosen is that of the Time-Triggered 
Architecture (TTA) \cite{kopetz:2001-22a}.  Time-triggered systems provide a number
of essential guarantees for safety-critical control systems designs.  In particular,
the TTA provides precise timing for periodic tasks, distributed fault-tolerance,
and replica determinism in redundant configurations.  These basic guarantees and
their implementations constitute some of the important high-level component 
services needed for our platform-based designs.  

\SubSection{Software architecture specification}
Simulink/Stateflow (SL/SF) models can be imported into a well-defined modeling format
that allows for analysis, extension, and code generation.  Graphical modeling
tools can read these models and perform software engineering design tasks.  
The SL/SF models are embedded in software components with well-defined
interfaces, and then mapped to well-defined distributed hardware models.

\SubSection{Code generation}
Model transfomations \cite{isis:great} can convert imported SL/SF models into a 
model representing an abstract syntax tree (AST) for C code fragments.   
Interpreters for the new AST model can create
code or directly perform simple static analyses such as checking variable 
initializations.  Generated C code is generic -- the tools currently support 
execution on a hardware implementation of the TTA (hardware available from TTTech\cite{TTTech})
or on a time-triggered virtual machine (VM) running on Linux (described below).

\SubSection{Scheduling}
Resource allocation in the TTA is controlled by a pre-generated cyclic schedule
created from task specifications and their communication dependencies.  We have 
created a simple schedule generation tool that uses the Gecode finite-domain constraint
programming library to search for cyclic schedules that meet the specifications.
Constraint models are an extension of earlier work in this area
\cite{sw:offlinescheduling}.

\SubSection{Modeling the execution platform}
The chosen time-triggered model of computation has been formalized using the DEVS 
formalism (Figure \ref{fig:devs_sys}) and simulated using the DEVS++ simulator \cite{DEVSpp}.  Simulation 
results for a time-triggered triple modular redundancy experiment were consistent 
with observed performance of a time-triggered implementation \cite{TK_TMR}.

\begin{figure} [h]
	\includegraphics[width=0.9\columnwidth]{DEVS_system}
	\caption{DEVS models for time-triggered virtual machine}
	\label{fig:devs_sys}
\end{figure}

\SubSection{Implementation of the execution platform}
In addition to tests on available time-triggered hardware, we have 
developed a portable time-triggered VM running on a networked cluster
of processors running standard Linux.  The portability of the VM allows the direct 
exploration of the capabilities and limitations of the 
services provided by the underlying operating system, and the effects of those 
limitations on the guarantees provided by the chosen MoC \cite{TK_TMR}.

